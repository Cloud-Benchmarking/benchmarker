%! Author = pseudo
%! Date = 12/18/2020

% Preamble
\documentclass[11pt]{article}

% Packages
\usepackage{amsmath}
\usepackage[colorinlistoftodos,textsize=tiny]{todonotes}

\usepackage{listings}
\usepackage{color}

\definecolor{dkgreen}{rgb}{0,0.6,0}
\definecolor{gray}{rgb}{0.5,0.5,0.5}
\definecolor{mauve}{rgb}{0.58,0,0.82}

\lstset{frame=tb,
  language=bash,
  aboveskip=3mm,
  belowskip=3mm,
  showstringspaces=false,
  columns=flexible,
  basicstyle={\small\ttfamily},
  numbers=none,
  numberstyle=\tiny\color{gray},
  keywordstyle=\color{blue},
  commentstyle=\color{dkgreen},
  stringstyle=\color{mauve},
  breaklines=true,
  breakatwhitespace=true,
  tabsize=3
}

% Document
\begin{document}
    \paragraph{Before the Big Kahuna}
    \begin{enumerate}
        \item before launch
        \begin{enumerate}
            \item GCP: asia-south1 / asia-northeast2 --> why not in regions json?
            \item Add date/time to each command for better tracing
            \item make sure every provider can handle 166+ nodes on one account
            \item Vultr is debating if they can give me more than 100
            \item Linode is at 50 currently
            \item Not sure where DigitalOcean is at
            \item have to look at GCP/Azure/AWS
            \item make sure to enable all ipv4 on vultr for api key
            \itm vultr needs higher traceroute timeout limits
            \item make script to check APIs to see if any `benchmark-` nodes exist (in case errors, need to manually delete, etc)
        \end{enumerate}
        \item quota limits to update
        \begin{enumerate}
            \item lame, my aws limit is 32 vCPUs (so 32 instances); gotta bump up those rookie numbers
            \item and 10 VMs/region for Azure
            \item and 25 VMs total for GCP
            \item oh, free trial for that, that makes sense for gcp
        \end{enumerate}
    \end{enumerate}
TODO: Error: Error launching source instance: Unsupported: Your requested instance type (t2.micro) is not supported in your requested Availability Zone (sa-east-1b). Please retry your request by not specifying an Availability Zone or choosing sa-east-1a, sa-east-1c.
	status code: 400, request id: d8847ccf-0ac6-48c0-93e3-b13ff4806bbe

    \paragraph{Benchmarking and Graphing Tools}
    \begin{enumerate}
        \item # drawing world map for "flight routes"
        \begin{enumerate}
            \item https://observablehq.com/@d3/world-airports
            \item https://observablehq.com/@d3/testing-projection-visibility?collection=@d3/d3-geo
            \item https://mbostock.github.io/d3/talk/20111116/airports.html
            \item https://vega.github.io/vega/tutorials/airports/
        \end{enumerate}

        \item # possible cloud providers
        \begin{enumerate}
            \item https://www.terraform.io/docs/providers/type/major-index.html
            \item https://www.terraform.io/docs/providers/type/cloud-index.html
            \item https://wikileaks.org/amazon-atlas/
        \end{enumerate}

        \item # some benchmarking resources
        \begin{enumerate}
            \item https://www.azurespeed.com/Azure/Latency
            \item https://cloud.google.com/blog/products/networking/perfkit-benchmarker-for-evaluating-cloud-network-performance
            \item https://github.com/GoogleCloudPlatform/PerfKitBenchmarker

            \item networking benchmarks
            \begin{enumerate}
                \item ping
                \item traceroute (to geolocate/plot the routes)
                \item iperf
                \item netperf
                \item https://openbenchmarking.org/suite/pts/network
            \end{enumerate}

            \item cpu benchmarks
            \begin{enumerate}
                \item spec cpu 2017
                \item https://openbenchmarking.org/suite/pts/single-threaded
            \end{enumerate}

            \item memory benchmarks
            \begin{enumerate}
                \item https://openbenchmarking.org/suite/pts/memory
            \end{enumerate}

            \item disk benchmarks
            \begin{enumerate}
                \item FIO
                \item https://openbenchmarking.org/suite/pts/compression
                \item https://openbenchmarking.org/suite/pts/disk
            \end{enumerate}

            \item misc benchmarks
            \begin{enumerate}
                \item https://openbenchmarking.org/suite/pts/server-cpu-tests
                \item https://openbenchmarking.org/suite/pts/server
            \end{enumerate}
        \end{enumerate}

        \item # geolocation tools
        \begin{enumerate}
            \item https://geoip2.readthedocs.io/en/latest/
            \item https://lite.ip2location.com/file-download
            \item https://github.com/maxmind/GeoIP2-python
            \item https://github.com/ip2location/IP2Location-Python/
            \item https://github.com/geopy/geopy
            \item Azure IP Ranges and Service Tags - Public Cloud : https://www.microsoft.com/en-us/download/details.aspx?id=56519
        \end{enumerate}
    \end{enumerate}

    \paragraph{Misc finds that inform benchmarking decisions}
    \begin{enumerate}
        \item https://twitter.com/linode/status/1252693011449950210?lang=en: After 60 seconds at 1500 IOPS there's a limit of 500 IOPS. Longer bursts are possible at lower I/O rates, and another burst is possible after enough time spent below 500 IOPS.
    \end{enumerate}

    \paragraph{ToDo}
    \begin{enumerate}
        \todo[color=green] real-time map of active routes under benchmark
        \todo real-time stats of active benchmarks
        \todo proxy terraform
        \todo programatically determine which regions AWS do not allow t2.micro in and ensure they're not added to the list
    \end{enumerate}

    \paragraph{Notes}
\begin{lstlisting}
[x]] should have run_ssh_benchmark_cmd that prepends `time` and captures that output
* collect metrics during benchmark (dstat, mstat, collectd, tcpdump?)
* sample.PercentileCalculator(throughputs, [50, 90, 99])

-=[ Phoronix Test Suite ]=-
curl http://phoronix-test-suite.com/releases/repo/pts.debian/files/phoronix-test-suite_10.0.1_all.deb --output pts.deb
apt install ~/pts.deb
$ phoronix-test-suite benchmark network
  Installing Test @ 14:48:55
  [...]
  takes too long to execute


-=[ COREMARK ]=-
# coremark (make PORT_DIR=linux64 ITERATIONS=1000000 XCFLAGS="-g -DUSE_PTHREAD -DPERFORMANCE_RUN=1")
mkdir ~/installs
curl --location https://github.com/eembc/coremark/archive/v1.01.tar.gz --output ~/installs/coremark.tar.gz
cd ~/installs
tar xvfz coremark.tar.gz
cd ~/installs/coremark-1.01/
make PORT_DIR=linux64 ITERATIONS=1000000 XCFLAGS="-g -DUSE_PTHREAD -DPERFORMANCE_RUN=1"
cat run*.log

-=[ GEEKBENCH 5]=-
# https://cdn.geekbench.com/Geekbench-4.2.0-Linux.tar.gz
curl --location https://cdn.geekbench.com/Geekbench-5.3.1-Linux.tar.gz --output ~/installs/geekbench.tar.gz
tar xvfz geekbench.tar.gz
~/Geekbench-5.3.1-Linux/geekbench5



curl --location https://cdn.geekbench.com/Geekbench-4.2.0-Linux.tar.gz --output ~/installs/geekbench.tar.gz
tar xvfz geekbench.tar.gz
~/Geekbench-4.2.1-Linux/geekbench4


curl --location https://cdn.geekbench.com/Geekbench-4.2.0-Linux.tar.gz --output ~/geekbench.tar.gz
tar xvfz ~/geekbench.tar.gz
~/Geekbench-4.2.0-Linux/geekbench4
\end{lstlisting}
\end{document}